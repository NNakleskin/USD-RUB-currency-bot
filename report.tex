\documentclass{article}
\usepackage{graphicx}
\usepackage{tikz}
\usepackage[russian]{babel}

\begin{document}
\title{Архитектура Telegram-бота для расчета курса доллара}
\author{Nakleskin Nikita}

\date{\today}
\maketitle

\section{Хранение данных}
Для хранения данных о курсах и комиссии будет использоваться Firebase, популярная облачная база данных. Firebase предоставляет надежное хранение данных и удобный API для взаимодействия с базой данных.

\section{Архитектура системы}
\begin{center}
\begin{tikzpicture}[node distance=2cm]
  \node (user) [rectangle, draw] {Пользователь};
  \node (telegram) [rectangle, draw, below of=user] {Telegram API};
  \node (bot) [rectangle, draw, below of=telegram] {Telegram-бот};
  \node (firebase) [rectangle, draw, below of=bot] {Firebase};
  \node (tinkoff) [rectangle, draw, left of=firebase, xshift=-3cm] {Тинькофф API};
  \node (cbr) [rectangle, draw, right of=firebase, xshift=3cm] {API ЦБ};
  
  \draw [->] (user) -- (telegram);
  \draw [->] (telegram) -- (bot);
  \draw [->] (bot) -- (firebase);
  \draw [->] (bot) -- (tinkoff);
  \draw [->] (bot) -- (cbr);
\end{tikzpicture}
\end{center}

Основные этапы обработки запросов:
\begin{enumerate}
  \item Пользователь отправляет запрос боту через Telegram API.
  \item Бот получает запрос и обращается к Тинькофф API и API ЦБ для получения актуальных данных о курсе доллара.
  \item Полученные данные сохраняются в Firebase. (При необходимости)
  \item Бот обрабатывает данные, учитывая заложенную комиссию, и отправляет пользователю ответ через Telegram API.
\end{enumerate}

\section{Управление ошибками}
Система будет использовать стандартные средства Python для управления ошибками. В случае недоступности источника данных или других неожиданных ситуаций, бот будет выдавать сообщения о временных технических неполадках и предлагать пользователю повторить запрос позже.

\section{Интеграция с Telegram API}
Бот будет взаимодействовать с Telegram API с помощью библиотеки aiogram, которая предоставляет удобный интерфейс для работы с Telegram ботами. Для аутентификации и обработки входящих запросов будут использоваться методы, предоставляемые aiogram.

\section{Отказоустойчивость системы}
Для обеспечения отказоустойчивости системы и продолжительной работы в случае сбоев будут предприняты следующие меры:
\begin{itemize}
  \item Регулярное резервное копирование данных в Firebase.
  \item Мониторинг доступности источников данных (Тинькофф API, API ЦБ) и автоматическое восстановление соединения при сбое.
  \item Реализация механизма повторной отправки запросов в случае ошибок соединения или других временных проблем.
\end{itemize}

\section{Масштабируемость и производительность}
Система должна быть масштабируемой и производительной для обработки запросов от множества пользователей. Для достижения этой цели можно применить следующие подходы:
\begin{itemize}
  \item Распределение нагрузки: Бот может быть запущен на нескольких серверах или использовать облачные сервисы для балансировки нагрузки и обработки большого количества запросов.
  \item Кэширование данных: Чтобы уменьшить нагрузку на источники данных, можно использовать механизм кэширования для временного хранения полученных данных о курсе и комиссии.
  \item Оптимизация запросов: Бот может выполнять запросы к источникам данных параллельно и использовать асинхронные операции для увеличения производительности.
\end{itemize}

\section{Интерфейс пользователя}
Интерфейс пользователя будет реализован с использованием кнопок и команд. Пользователь сможет взаимодействовать с ботом, нажимая на кнопки, чтобы получить актуальный курс доллара или выполнить другие команды. Кнопки будут отображаться в чате с ботом, и пользователь сможет выбирать нужные опции с помощью нажатия на кнопки.
\end{document}